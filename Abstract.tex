\documentclass{article}

\begin{document}
\title{Jonathan Borowski}
$  $\chapter{Abstract}\\
In the following game analysis, The fighting game Rivals of Aether for PC will be analyzed with regards to it's contents, it's background, it's success as well as it's reception.\\
The analysis will heavily focus on the game elements, their origin and relation to the super smash bros. franchise. During the analysis, the differing game modes offered will be played, all characters will be tested, and the game will be learned as it is taught through it's tutorial. During this, the game mechanics, design, plot and art choices will be reflected upon.\\
In order to understand and explain the game's inception, the background of the independent game developer Dan Fornace will be researched, along with the state of the genre following up on the creation, and the feedback of it's users during development.\\
The state of the game and it's success will be evaluated based on it's community on social media and reception, as well as commercial acclaim. Those findings will be compared to other games of the genre.\\Lastly, the game's pro scene will be evaluated in comparison to it's competitors.\\
\newpage
\chapter{Introduction}\\
This introduciton should have the following:\\
\textbf{Historical reflection of the genre} \cite{jeffdunn2012:1}\\
While the fighting genre rose to popularity during the early 90's, it has actually been introduced quite a while back. Sega's 1976 Heavyweight Champion features a design choice that influences the genre today: a side-view perspective. This would manifest itself in many fighter games well into the era of 3D graphics.\\
In 1984, Technos Japan's Karate Champ has released what is considered to be the first fighting game as we have come to know them. Reason: it features dual-joystick controls, multiple stages, a score system, and a martial arts inspired theme.\\
The true rise to popularity however, begins in 1991 with the release of Capcom's Street Fighter II. Notable features: tight controls, up to date graphics, life bars. It's innovations include a wide variety of characters and movesets, and versus play.\\
The next notable followed the next year: 1992 marked the debut of Midway's Mortal Kombat. This title did not only bring it's own innovations to the table, taking the form of gory fatalities and digital character models, it was also criticized heavily for it's violence and bloody effects.\\
Enter 1999: The release of Super Smash Bros. for the Nintendo 64. During a time in which the fighting game market was over saturated with sequels, offering repetition rather than new experiences, this title changed the formula: No life bars, emphasis on ring out over KO, large interactive and large stages, an all star cast of Nintendo's characters.\\
Most notable are not only the easy to play but hard to master difficulty, which captivated casuals and dedicated players alike, but the large scale community that has formed around it.\\
Following is a decline, then a rise to popularity by smash, cover it, then proceed to the genre today. note that smash is the founder of a subgenre.\\
-The actual present day state of the genre-\\

-Why I chose to analyze Rivals of Aether-\\
Rivals of Aether is unique take on the 4 Player Battle Royale genre. It sets itself apart by being an indie Title: it features pixel art style, matching retro music, and a cast of fighters which is heavily balanced and diverse. The game features online competitive play as it's forte. The reason I chose this title for the Games Analysis is because I believe it is a breath of fresh air to a genre which I enjoy a lot. It has also gained a lot of traction during it's development cycle [reference], causing the hype to build up around it preceding launch, which has formed a community around it.\\
-Exact definition of the focus of this analysis-\\
This analysis will contain three major chapters:\\
1) The background of the game, it's competitor's and it's creation.\\
2) In depth analysis of the game itself: Game features, modes, art choices, character moves, the narrative and more.\\
3) Community & Reception: The competitive scene, the traction the title has gained, comparison to it's competitors, critical acclaim.\\

\newpage
\chapter{Analysis}\\
Beginning of the actual analysis\
Game version analyzed: 1.2.4\\%set version
\textbf{Overview of the game's core mechanics}\\
Rivals of Aether does not feature a health bar, nor does it count the amount of damage until one's defeat. Instead, similar to any smash bros. game, it counts damage received as a multiplier, taking affect when being launched by an attack. Simply put: attack moves do both damage and also launch their victim. Some moves will be more damaging than launching, some exactly the opposite.\
Upon leaving the camera's boundaries, the player will either lose a life or lose score, depending on the game mode. Thus, each player's objective in any standard game mode, is launching his opponent's off the screen, causing them to lose points.\
 Additionally, the amount of players able to play in a match is 4, as opposed to the traditional duel set up found in most fighting games. These 4 players may be pitted against each other in all sorts of manners: free for all, team fight of any arrangement. Any of the four may also be controlled by an AI with varying difficulty. The game is 2D: meaning vertical movement counts as jumping and special moves in that direction, whilst horizontal movement is possible through walking and running, similar to a platform game.\\ %expand on this

Here is a summation of what a player can do in the game, followed by the required input:\
Walk in varying speeds, run, and duck, all executable through a control stick or arrow keys.\
Jump: using the jump button. By default, a character can jump twice. Additionally, each character may also jump using a wall while moving into it. This is a crucial movement as it is used to recover after being launched by an opponent's attack.\
Attack: a move which is executed by pressing the attack button, and is context sensitive to whatever the character is doing. In a neutral state, meaning that the character is standing still, this attack is a jab combo. If, however, a direction input is present while attacking, this attack will become a tilt attack in that direction. These options then double themselves when taking the following into account: attacks can also be executed while airborne. During this state, the attack button functions just like when standing, having a neutral state attack and one for any direction input. These moves deal damage to victims, and launches them.\\
Special Moves: these moves differ from attack moves. They will execute in the same manner while grounded and airborne. However, these moves are special; those contain the projectiles, explosions, and other non hand to hand combat moves. They come in different directions and neutral. The up special moves also count as recovery moves, such as additional jumps and platform generation. Special moves deal damage and launch enemies.\\
Dodge: The dodge button, used while idle and grounded, will execute a parry. This is a counter move to incoming attacks and special moves. Upon a successful parry, the player becomes invincible for a short amount of time, and the opponent is stunned as a punishment, if the attack was in melee range and not a jab attack. This is an advanced technique which is crucial when facing constant aggression. A well timed parry allows a strong counter attack.\
When used with a horizontal direction input, the player executes a roll, which is a side stepping dodge. When used while airborne, an air dodge is used, which may be static or directed. Any of the dodges will grant short invulnerability. The air dodge may be used as a last effort, after all jumps have been spent, in order to safely return to the arena.\\
Strong Attack: This move is a single attack which aims to launch an opponent. These are slow in comparison to other moves, and while they do deal damage, stringing smaller strikes together will yield better and safer results. However, no move is better at killing then strong attacks. If the attack connects with an opponent who has suffered significant damage, he will surely be knocked out. These moves may only be used on the ground, they may be charged in order to increase the force of the strike, and launch enemies further.\\

\chapter{Overview of the characters}

Characters in the game are:\
Zetterburn, Orcane, Wrastor, Kragg, Forsburn, Maypul, Absa, Etalus, Ori, Ranno, Clairen %set characters, explain moves, hint at balance

\textbf{Zetterburn}\\
Zetterburn is a medium weight, fire lion themed combatant. The ability to apply a burning effect on his foes, which delivers small amounts of damage over time, is unique to this character. This fighter's special moves apply said burning effect, through means of ignition, fire projectiles, explosions and self ignition.\\
\textbf{Kragg}\\
Kragg is a heavy weight, earth beetle themed combatant. He is also slow and doesn't combo as quickly as his opponents. However, what he lacks in speed, he makes up for in power and defense. His specials deal damage using his element, be it by creating earth spikes from the ground or creating and throwing cubed earth. These cubes may be broken by any character, sending it's fragments flying in the corresponding direction and dealing damage.\\
\textbf{Orcane}\\
This Hybrid between a cat and a killer whale is the title's water themed fighter. Orcane is a swift combo oriented character, with an interesting mechanic: His special moves sends out a splash of water that sticks upon landing. From this point on, Orcane may interact with that water: shoot some bubbles out of it to deal damage, or teleport to it either to launch a foe or to recover back to the stage.\\
\textbf{Wrastor}\\
Wrasot is a bird. He is air themed and, as such, is a very light-weight and mobile fighter. His side special will throw a wind current, which speeds up his movement through it, in a horizontal line. Unlike other characters, Wrastor can jump up to four times, and is only able to unleash a strong attack from the air (as opposed to all others, who have to be grounded).
\\
\textbf{Maypul}\\
A weasel-esque forest dweller, Maypul is the forest themed fighter. She uses vines and plants as an offensive means as well as recovery. When she applies her watcher's mark to a target, she is able to quickly jump to it with her up special. Other than that, she is a very agile but light weight combatant.
\\
\textbf{Forsburn}\\
This dagger wielding lion, who is Zetterburn's brother, is a smoke themed combatant. His special moves allow him to deploy a smoke screen, to spawn an illusion of himself, and to disappear in an explosions. His smoke also has a second purpose, other than a decoy: Once three clouds of smoke have been absorbed using his down special move, Forsburn becomes charged, and can combust to launch opponents and damage them.
\\
\textbf{Etalus}\\
This metal jaw wielding polar bear is the game's ice fighter. Etalus packs a heavy weight, and a unique mechanic: Ice and ice armor. Some of Etalus' special attacks will cause hail and ice to cover the stage, upon which Etalus can slide and increase his mobility. Another special move allows him to break the ice around him in order to equip ice armor. This makes him resilient to launching attacks. The armor will break after sufficient damage has been dealt, or through using his down special to smash it into the ground, shattering the armor.
\\
\textbf{Absa}\\
Wield the power of lighting with this ram resembling fighter. Absa's side special will create a a cloud, which has a multitude of uses: zap opponents when appearing, conduct a large lighting between Absa and the cloud or cause it to explode with thunder. Absa is a light weight ranged specialist, which can combo many attacks in rapid succession.
\\
Ori \& Sein\\
\\
Ranno\\
\\
Clairen\\
\\

\chapter{Stages}

About stages in the game: Each and every fight must take place on one of the following stages. These stages have two modes of play, akin to Nintendo's Smash bros. for WiiU nad 3DS: Normal and Aether variant. The difference between the modes is that the normal ones are almost completely static, which is suitable for high levels of play. The Aether variants will make each stage more lively, with breakable objects, springs, and thunder. This welcomes new and fun mechanics for players to take advantage of. The Aether variants are not available to play during online matches of all kinds.\\

\textbf{Tower of heaven}\\
Take the fight to a tower high above the clouds. Above a large flat base, which is common to all normal mode stages, are three smaller platforms, with the outer ones being golden and of the same height. The inner one is slightly elevated, shaping a triangle. Another side about this stage: above the central platform is a window, which remains empty during most of the time. Every couple of seconds though, a small green character walks up to it from inside the tower, stands still, and then wanders off. During this, if the character is attacked, he explodes and reins down green particles.
Aether mode: the base has an elevated center, and some butterflies are flying on the high horizontal edges. The true change however, is a book which appears in the center of the stage every once in a while during a fight. The first character to move into it will unleash a random rule which applies to all other fighters. The rules are as such:\\
\begin{itemize}
\item Law # 1: Thou shall not touch the golden platforms.
\item Law # 2: Thou shall not touch walls from the sides.
\item Law # 3: Thou shall not Parry or Dodge.
\item Law # 4: Thou shall not touch a living thing (which is where the butterflies, the grass, and the green character play a role).
\end{itemize}
\\

\textbf{Fire Capitol}\\
A static stage atop a flat roofed building, U shaped with some crates fighters can stand on. There's just not that much else to say about this one.\
Aether mode: The stage is arranged differently,so that the arena is asymmetric, but is still completely static.\\
\textbf{Frozen Fortress}\\
This frozen playground features a flat base, a large central platform, and some smaller ones on the upper corners. The ice is slippery, and the stage is surrounded by water, although that's just for the effects. Unfortunately, swimming is not possible.\\
Aether mode: The sides of the base are walled of by some ice, preventing knock outs to that direction (Only the top of the screen is available for kills). Attacking the ice will cause it to shatter, giving way to launches and recoveries. A chain is revealed in the background and, after a set period of time, will pull in a fresh ice wall.
\\
\textbf{Endless Abyss}\\
This stage is a large chunk of rock falling down an abyss. It has a large base and no additional platforms. As it's name suggests, the fall is endless. As such, the stage is completely static and nothing happens, which allows for uninterrupted fighting.
Aether mode: %expand
\\
\textbf{Aethereal Gates}\\

\\
\textbf{Merchant Port}\\
Fight in the docks of a port. The docks have many smaller platforms and crates on which the fighters can position themselves, making way for vertical attacking options. This stage is also surrounded by water.\\
Aether mode: This version features a Ferris wheel in the center of the stage, and two hydrants, to the horizontal ends of the base. Attacking a hydrant will cause it to jet a stream of water into the skies, which deals damage and launches a player from the hydrant, making for a potent finisher. The hydrants will recharge and be active again upon completion. The current state of the hydrant is telegraphed by a vertical loading bar on it.
\\
\textbf{Air Armada}\\
Fight on a sky ship in the air. The ship's deck serves as the base, while two smaller platforms are available at the horizontal ends of the stage. Being able to be under the stage for recovery makes this stage unique.\\
Aether mode: A large spring is present at the ship's deck, which amplifies the jumping power of any fighter jumping from it. This in turn will make even the heaviest fighters very mobile.
\\
\textbf{Rock Wall}\\
Fight atop a stone wall. The base of the wall is wide, and two pillars with two floors each rise from it's sides.\\
Aether Mode: The wall completely covers the bottom of the screen. This means no kill are available when launching in this direction. In this mode, however, the pillars can be broken if attacked, which will create a hole in the wall from which fighters may fall.\\
\\
Spirit Tree\\%todo
\\
\textbf{Blazing Hideout}\\
This stage is a flat burning rooftop. In normal mode, the fire and smoke is just a visual effect. This stage has a large platform above it's base, making fighting possible on the elevated height or down on the roof top.
Aether mode: The fire rises and harms players. Unsuspecting fighters will be consumed be the fire.
\\
\textbf{Treetop Lodge}\\
Treetop Lodge is a tree house, with a large flat base and two platforms of differing heights. This stage is static but asymmetric.
Aether mode: A giant plant will snag fighters from the horizaontal ends of the stage. It reaches up from the bottom of the screen and devours the player, costing a life.
\\
\textbf{Tempest Peak}\\
A stormy peak of a great mountain, this stage features a large base and four smaller  platforms. These platforms are higher than the base, with the outer pair only slightly higher and the inner pair being above that. A statue rests upon each of these platforms.
Aether mode: The smaller platforms are now moving. Each pair, outer and inner, will remain at the same height at all times, but they move vertically, pause, and return to their original position. The inner platforms will merge with the ground. In this mode, the statues will glow. This warns fighters that lighting is about to strike this statue. Players will receive great damage when touching the lighting, and will be knocked away from it, potentially killing them.
\\

\chapter{Game modes}\\


\textbf{Single Player Modes}\\ %expand. note the online play and the tutorial, also

Versus Mode\\
This is the local multiplayer mode, in which up to 4 characters can fight in any arrangement. The fight takes place on a chosen stage, with either normal or Aether variants available. A settings menu for this mode allows for some changes, like disabling pause, adjusting knock back applied, and playing in a tourney format.
\\

Story Mode\\
Fight in a sequence of battles, separated by narrative. This mode tells the story of each of the characters. Any story will have the chosen character fight all of the original six cast, and will end the story with a purple colored variant of that story's villain. There is no narrative justification for most of the fights, other then a few exceptions and the final showdown with the villain.
TODO: Cover each plot.
\\

Abyss Mode\\

Practice Mode\\
This serves as a controlled environment. Play against any number of opponents, doing whatever you wish. This mode allows to practice while featuring control of the game speed, and the telegraphing of character's hitboxes, and the sweet spot for attack animations.\\

Tutorial\\
This interactive tutorial will teach any newcomer how to play the game. It is segmented into beginner, intermediate, and advanced, with each of these being segmented further into movement, defense and offense. While the beginner ones are pretty basic and are relatively easy to learn, the game doesn't hold back on it's players when it comes to learning advanced techniques. Vectoring, wave dashing and jump canceling are all well known advanced maneuvers from the smash series \cite{atomic2017:2} \cite{manual:3}, %SITE
but none of the games will teach the players anything about it. In Rivals, however, advanced techniques are front and center in this tutorial.\\
The game also features character specific tutorials for each of the characters released. These tutorials will teach about the character's signature moves, and how to use them correctly. Note: The tutorials are available for all characters from the start, even for unlocked ones. At the time of writing, I do not own Ori, Ranno and Clairen, yet their tutorials are available to play.\\

\textbf{Online Modes}\\
Features: Ranked match mode, in which players duel for rank and online fame. During matchmaking, players view their match's region (As in, EU / US), as well as said match's ping. Both players may then accept or decline their match. If one declines, both will be matched to other players. If both agree, then the match can begin. During this, a character may be selected, controls may be set up, and the character may even be tested. More on that on the character selection screen.\\
Once the match has been accepted and both characters have been chosen, begins the process of \textbf{stage elimination}. From all of the available stages, players take turn striking a stage. This stage is taken out, and the process resumes until only a single stage remains, in which the fight will take place.\\
A match is a best of 3 rounds, with each round being a 3 stock survival match. After a round in the match, if another round follows, the winner gets to strike two stages, and the loser gets to chose the next stage from the available ones.\\
Upon victory of a match, in game currency is earned, and the rank is increased. Upon defeat, rank is decreased.\\

Exhibition mode:\\
Plays similarly to ranked mode, with a handful of notable differences: The first stage is chosen at random, and players are matched regardless of their approval or not. They may still back out if the latency is to high, for example, before the match begins. Subsequent stages are chosen identically to ranked mode. No ranks are involved, and no in game currency is earned.\\
Players may challenge their opponent to a money match if the corresponding button is pressed. This match is the same but accepts a wager of in game currency from both players. It is initiated only if the other player accepts the offer.\\

Friendly Match\\
Play a custom game with friends. This mode allows for complete control over it's settings and stage selection, as it is in the local versus mode.
\\
Team Fight\\
Fight in a 2 Vs. 2 match, either with a local second player or an AI controlled ally. Fight an invited friend, or be matched against an enemy team similarly to Exhibition mode.
\\

Online mode also offers a view of the player's statistics, like winrate and most frequented character, as well as the leader board, where one can see who's at the top, what's his position on the ladder and where his friends are placed. Region can be set here.\\


\chapter{Featured art style and music}

\textbf{Visual art style}\\
Rivals of Aether sets itself apart from other games in the genre by being fully 2D, as opposed to it's 2.5D competition, and features pixel art graphics. This art style, while it may or may not be to the consumers liking, does offer some points of interest unique to a game as such. It allows the fighter game to have pixel perfect accuracy, well telegraphed hit boxes and attack sweet spots. The tutorial makes use of these while teaching the corresponding subjects in game, with hit boxes being visualized as a green box, and the sweet spots appearing as red dots on the attack.
While playing the practice mode, the player may chose to display either of these settings.\\

Although characters boast a multitude of different color pallets, players can actually create their own custom colors for each character, and then take those into battle.\\

\textbf{Music}\\
The music featured in the game is kept to it's retro aesthetics. It's completely digital and instrumental. Each stage has it's own musical score, and the winning fighter is rewarded with his corresponding stage's tune once the victory stage is shown.\\


\chapter{Reception}

%SITES

The games reception can be summarized as positive \cite{steam:6}.\\
During the time of this review, Steam's reception is as follows:\
85% out of 119 recent user reviews have been positive.\
89% of the 3,289 of the user reviews, recent or not, have been positive.\


Here are some more sites and their respective review scores: \cite{metacritic:7} %site this.

\begin{itemize}
\item IGN Italy 8.1
\item Destructoid 9
\item Metacritic 8.3 on the user reviews based on 11 ratings.
\end{itemize}

\chapter{Community}

%the reddit
\textbf{reddit}\\
Rival's reddit page boasts over 1,200 users, compared to Super Smash Bros. reddit of about 243,400, and Brawlhalla's reddit of 18,500 users \cite{reddit:4}.\\

\textbf{Twitch}\\
On Twitch, Rivals of Aether some 11,740 followers, compared to Super Smash bros. 29,260, 336,830, 20,990, 450,160 viewers for the series' entries of the original, Melee, Brawl, and the Wii U version respectively. Brawlhalla has got about 69,320 followers on Twitch. \cite{twitch:5}\\

\textbf{facebook}\\

TODO\\

\chapter{Pro Scene}

TODO\\

\newpage
\bibliography{Bib}
\bibliographystyle{ieeetr} 

\end{document}