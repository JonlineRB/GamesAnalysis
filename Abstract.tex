\documentclass[11pt]{article}
\usepackage{cite}

\begin{document}
\title{Jonathan Borowski}
\chapter{Abstract}\\
In the following game analysis, The fighting game Rivals of Aether for PC will be analyzed with regards to it's contents, it's background, it's success as well as it's reception.\\
The analysis will heavily focus on the game elements, their origin and relation to the super smash bros. franchise. During the analysis, the differing game modes offered will be played, all characters will be tested, and the game will be learned as it is taught through it's tutorial. During this, the game mechanics, design, plot and art choices will be reflected upon.\\
In order to understand and explain the game's inception, the background of the independent game developer Dan Fornace will be researched, along with the state of the genre following up on the creation, and the feedback of it's users during development.\\
The state of the game and it's success will be evaluated based on it's community on social media and reception, as well as commercial acclaim. Those findings will be compared to other games of the genre.\\Lastly, the game's pro scene will be evaluated in comparison to it's competitors.\\
\newpage
\chapter{Introduction}
This introduciton should have the following:\\
-Historical reflection of the genre NOTE: cite gamesradar-\\
While the fighting genre rose to popularity during the early 90's, it has actually been introduced quite a while back. Sega's 1976 Heavyweight Champion features a design choice that influences the genre today: a side-view perspective. This would manifest itself in many fighter games well into the era of 3D graphics.\\
In 1984, Technos Japan's Karate Champ has released what is considered to be the first fighting game as we have come to know them. Reason: it features dual-joystick controls, multiple stages, a score system, and a martial arts inspired theme.\\
The true rise to popularity however, begins in 1991 with the release of Capcom's Street Fighter II. Notable features: tight controls, up to date graphics, life bars. It's innovations include a wide variety of characters and movesets, and versus play.\\
The next notable followed the next year: 1992 marked the debut of Midway's Mortal Kombat. This title did not only bring it's own innovations to the table, taking the form of gory fatalities and digital character models, it was also criticized heavily for it's violence and bloody effects.\\
Enter 1999: The release of Super Smash Bros. for the Nintendo 64. During a time in which the fighting game market was over saturated with sequels, offering repetition rather than new experiences, this title changed the formula: No life bars, emphasis on ring out over KO, large interactive and large stages, an all star cast of Nintendo's characters.\\
Most notable are not only the easy to play but hard to master difficulty, which captivated casuals and dedicated players alike, but the large scale community that has formed around it.\\
Following is a decline, then a rise to popularity by smash, cover it, then proceed to the genre today. note that smash is the founder of a subgenre.\\
-The actual present day state of the genre-\\
-Why I chose to analyze Rivals of Aether-\\
Rivals of Aether is unique take on the 4 Player Battle Royale genre. It sets itself apart by being an indie Title: it features pixel art style, matching retro music, and a cast of fighters which is heavily balanced and diverse. The game features online competitive play as it's forte. The reason I chose this title for the Games Analysis is because I believe it is a breath of fresh air to a genre which I enjoy a lot. It has also gained a lot of traction during it's development cycle [reference], causing the hype to build up around it preceding launch, which has formed a community around it.\\
-Exact definition of the focus of this analysis-\\
This analysis will contain three major chapters:\\
1) The background of the game, it's competitor's and it's creation.\\
2) In depth analysis of the game itself: Game features, modes, art choices, character moves, the narrative and more.\\
3) Community & Reception: The competitive scene, the traction the title has gained, comparison to it's competitors, critical acclaim.\\

\bibliography{Bib}{}
\bibliographystyle{plain}

\end{document}