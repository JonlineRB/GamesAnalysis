\documentclass[11pt]{article}
\usepackage{cite}

\begin{document}
\title{Jonathan Borowski}
$  $\chapter{Abstract}\\
In the following game analysis, The fighting game Rivals of Aether for PC will be analyzed with regards to it's contents, it's background, it's success as well as it's reception.\\
The analysis will heavily focus on the game elements, their origin and relation to the super smash bros. franchise. During the analysis, the differing game modes offered will be played, all characters will be tested, and the game will be learned as it is taught through it's tutorial. During this, the game mechanics, design, plot and art choices will be reflected upon.\\
In order to understand and explain the game's inception, the background of the independent game developer Dan Fornace will be researched, along with the state of the genre following up on the creation, and the feedback of it's users during development.\\
The state of the game and it's success will be evaluated based on it's community on social media and reception, as well as commercial acclaim. Those findings will be compared to other games of the genre.\\Lastly, the game's pro scene will be evaluated in comparison to it's competitors.\\
\newpage
\chapter{Introduction}\\
This introduciton should have the following:\\
-Historical reflection of the genre NOTE: cite gamesradar-\\
While the fighting genre rose to popularity during the early 90's, it has actually been introduced quite a while back. Sega's 1976 Heavyweight Champion features a design choice that influences the genre today: a side-view perspective. This would manifest itself in many fighter games well into the era of 3D graphics.\\
In 1984, Technos Japan's Karate Champ has released what is considered to be the first fighting game as we have come to know them. Reason: it features dual-joystick controls, multiple stages, a score system, and a martial arts inspired theme.\\
The true rise to popularity however, begins in 1991 with the release of Capcom's Street Fighter II. Notable features: tight controls, up to date graphics, life bars. It's innovations include a wide variety of characters and movesets, and versus play.\\
The next notable followed the next year: 1992 marked the debut of Midway's Mortal Kombat. This title did not only bring it's own innovations to the table, taking the form of gory fatalities and digital character models, it was also criticized heavily for it's violence and bloody effects.\\
Enter 1999: The release of Super Smash Bros. for the Nintendo 64. During a time in which the fighting game market was over saturated with sequels, offering repetition rather than new experiences, this title changed the formula: No life bars, emphasis on ring out over KO, large interactive and large stages, an all star cast of Nintendo's characters.\\
Most notable are not only the easy to play but hard to master difficulty, which captivated casuals and dedicated players alike, but the large scale community that has formed around it.\\
Following is a decline, then a rise to popularity by smash, cover it, then proceed to the genre today. note that smash is the founder of a subgenre.\\
-The actual present day state of the genre-\\
-Why I chose to analyze Rivals of Aether-\\
Rivals of Aether is unique take on the 4 Player Battle Royale genre. It sets itself apart by being an indie Title: it features pixel art style, matching retro music, and a cast of fighters which is heavily balanced and diverse. The game features online competitive play as it's forte. The reason I chose this title for the Games Analysis is because I believe it is a breath of fresh air to a genre which I enjoy a lot. It has also gained a lot of traction during it's development cycle [reference], causing the hype to build up around it preceding launch, which has formed a community around it.\\
-Exact definition of the focus of this analysis-\\
This analysis will contain three major chapters:\\
1) The background of the game, it's competitor's and it's creation.\\
2) In depth analysis of the game itself: Game features, modes, art choices, character moves, the narrative and more.\\
3) Community & Reception: The competitive scene, the traction the title has gained, comparison to it's competitors, critical acclaim.\\

\newpage
\chapter{Analysis}\\
Beginning of the actual analysis\
Game version analyzed: %set version
Overview of the game's core mechanics\\
Rivals of Aether does not feature a health bar, nor does it count the amount of damage until one's defeat. Instead, similar to any smash bros. game, it counts damage received as a multiplier, taking affect when being launched by an attack. Simply put: attack moves do both damage and also launch their victim. Some moves will be more damaging than launching, some exactly the opposite.\
Upon leaving the camera's boundaries, the player will either lose a life or lose score, depending on the game mode. Thus, each player's objective in any standard game mode, is launching his opponent's off the screen, causing them to lose points.\
 Additionally, the amount of players able to play in a match is 4, as opposed to the traditional duel set up found in most fighting games. These 4 players may be pitted against each other in all sorts of manners: free for all, team fight of any arrangement. Any of the four may also be controlled by an AI with varying difficulty. The game is 2D: meaning vertical movement counts as jumping and special moves in that direction, whilst horizontal movement is possible through walking and running, similar to a platform game.\\ %expand on this

Here is a summation of what a player can do in the game, followed by the required input on a controller:\
Walk in varying speeds, run, and duck, all executable through a control stick or arrow keys.\
Jump: using the jump button. By default, a character can jump twice. Additionally, each character may also jump using a wall while moving into it. This is a crucial movement as it is used to recover after being launched by an opponent's attack.\
Attack: a move which is executed by pressing the attack button, and is context sensitive to whatever the character is doing. In a neutral state, meaning that the character is standing still, this attack is a jab combo. If, however, a direction input is present while attacking, this attack will become a tilt attack in that direction. These options then double themselves when taking the following into account: attacks can also be executed while airborne. During this state, the attack button functions just like when standing, having a neutral state attack and one for any direction input. These moves deal damage to victims, and launches them.\\
Special Moves: these moves differ from attack moves. They will execute in the same manner while grounded and airborne. However, these moves are special; those contain the projectiles, explosions, and other non hand to hand combat moves. They come in different directions and neutral. The up special moves also count as recovery moves, such as additional jumps and platform generation. Special moves deal damage and launch enemies.\\
Dodge: The dodge button, used while idle and grounded, will execute a parry. This is a counter move to incoming attacks and special moves. Upon a successful parry, the player becomes invincible for a short amount of time, and the opponent is stunned as a punishment, if the attack was in melee range and not a jab attack.\
When used with a horizontal direction input, the player executes a roll, which is a side stepping dodge. When used while airborne, an air dodge is used, which may be static or directed, in any direction. Any of the dodges will grant short invulnerability.\\



Game modes: Online, Local\ %expand. note the online play and the tutorial, also story mode


Characters in the game are:\
Zetterburn, Orcane, Wrastor, Kragg, Forsburn, Maypul, Absa, Etalus, Ori, Ranno, Clairen %set characters, explain moves, hint at balance

Maps of the game: \ %

featured art style and music\\ %

\bibliography{Bib}{}
\bibliographystyle{plain}

\end{document}